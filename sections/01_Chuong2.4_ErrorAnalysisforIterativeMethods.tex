\section{Phân tích sai số cho các phương pháp lặp (Error Analysis for Iterative Methods)}

Trong phần này, ta khảo sát \textit{bậc hội tụ} của các sơ đồ lặp hàm, 
từ đó rút ra phương pháp Newton như một công cụ để đạt hội tụ nhanh. 
Đồng thời, ta cũng xem xét các cách cải thiện tốc độ hội tụ của Newton trong một số trường hợp đặc biệt.

Trước hết, cần một khái niệm để đo tốc độ hội tụ của một dãy.

\subsection*{Định nghĩa 2.7 (Bậc hội tụ)}

Giả sử $\{p_n\}$ là một dãy hội tụ về $p$, với $p_n \neq p$ với mọi $n$. 
Nếu tồn tại các hằng số dương $\lambda$ và $\alpha$ sao cho
\[
\lim_{n \to \infty} \frac{|p_{n+1} - p|}{|p_n - p|^\alpha} = \lambda,
\]
thì ta nói dãy $\{p_n\}$ hội tụ về $p$ với \textbf{bậc $\alpha$}, 
và $\lambda$ được gọi là \textit{hằng số sai số tiệm cận}.

Một kỹ thuật lặp dạng $p_n = g(p_{n-1})$ được gọi là \textit{hội tụ bậc $\alpha$} 
nếu dãy $\{p_n\}$ hội tụ về nghiệm $p = g(p)$ với bậc $\alpha$.

Nói chung, bậc hội tụ càng cao thì dãy càng nhanh tiến tới nghiệm. 
Hằng số $\lambda$ cũng ảnh hưởng đến tốc độ, nhưng yếu tố quyết định là bậc $\alpha$. 
Hai trường hợp thường gặp:
\begin{enumerate}
  \item Nếu $\alpha = 1$ và $\lambda < 1$, dãy hội tụ tuyến tính (\textit{linear convergence}).
  \item Nếu $\alpha = 2$, dãy hội tụ bậc hai (\textit{quadratic convergence}).
\end{enumerate}

Ví dụ minh hoạ tiếp theo sẽ so sánh một dãy hội tụ tuyến tính với một dãy hội tụ bậc hai, 
cho thấy lý do tại sao nên tìm các phương pháp có bậc hội tụ cao.

\subsection*{\textbf{Minh hoạ}}

Giả sử $\{p_n\}$ là dãy hội tụ tuyến tính về $0$ với hằng số sai số tiệm cận $0.5$, tức là
\[
\lim_{n \to \infty} \frac{|p_{n+1}|}{|p_n|} = 0.5.
\]

Đồng thời, xét một dãy $\{q_n\}$ hội tụ bậc hai về $0$ với cùng hằng số sai số tiệm cận $0.5$, tức là
\[
\lim_{n \to \infty} \frac{|q_{n+1}|}{|q_n|^2} = 0.5.
\]

Với giả thiết $|p_0| = |q_0| = 1$, Bảng~\ref{tab:conv-rates} so sánh tốc độ hội tụ của hai dãy.

\begin{center}
\captionof{table}{So sánh tốc độ hội tụ tuyến tính và bậc hai}
\label{tab:conv-rates}
\begin{tabular}{|c|c|c|}
\hline
$n$ & Tuyến tính: $(0.5)^n$ & Bậc hai: $(0.5)^{2^n - 1}$ \\
\hline
1 & $5.0000 \times 10^{-1}$ & $5.0000 \times 10^{-1}$ \\
2 & $2.5000 \times 10^{-1}$ & $1.2500 \times 10^{-1}$ \\
3 & $1.2500 \times 10^{-1}$ & $7.8125 \times 10^{-3}$ \\
4 & $6.2500 \times 10^{-2}$ & $3.0518 \times 10^{-5}$ \\
5 & $3.1250 \times 10^{-2}$ & $4.6566 \times 10^{-10}$ \\
6 & $1.5625 \times 10^{-2}$ & $1.0842 \times 10^{-19}$ \\
7 & $7.8125 \times 10^{-3}$ & $5.8775 \times 10^{-39}$ \\
\hline
\end{tabular}
\end{center}

Ta thấy dãy hội tụ bậc hai đạt độ chính xác $10^{-38}$ chỉ sau 7 bước, 
trong khi dãy hội tụ tuyến tính cần ít nhất 126 bước để đạt cùng mức chính xác.


\subsection*{\textbf{Định lý 2.8}}

Giả sử $g \in C[a,b]$ và $g$ ánh xạ $[a,b]$ vào chính nó. 
Giả sử thêm rằng $g'$ liên tục trên $(a,b)$ và tồn tại hằng số $k < 1$ sao cho
\[
|g'(x)| \leq k, \quad \forall x \in (a,b).
\]

Nếu $g'(p) \neq 0$, thì với mọi giá trị khởi đầu $p_0 \neq p$ trong $[a,b]$, 
dãy
\[
p_n = g(p_{n-1}), \quad n > 1,
\]
hội tụ tuyến tính về nghiệm duy nhất $p \in [a,b]$, 
với hằng số sai số tiệm cận $|g'(p)|$.

\subsection*{\textbf{Định lý 2.9}}

Giả sử $p$ là nghiệm của phương trình $x = g(x)$. 
Nếu $g'(p) = 0$ và $g''$ liên tục với $|g''(x)| < M$ trên một khoảng mở $I$ chứa $p$, 
thì tồn tại $\delta > 0$ sao cho, với mọi $p_0 \in [p-\delta,\, p+\delta]$, 
dãy
\[
p_n = g(p_{n-1}), \quad n > 1,
\]
hội tụ ít nhất bậc hai đến $p$. Hơn nữa, với $n$ đủ lớn,
\[
|p_{n+1} - p| \;\leq\; \tfrac{M}{2}\,|p_n - p|^2.
\]


\subsection*{\textbf{Multiple Roots}}

Trong các phần trước, ta đã giả định rằng $f'(p) \neq 0$, 
với $p$ là nghiệm của phương trình $f(x) = 0$. 
Tuy nhiên, trong thực tế, khi $f'(p) = 0$ đồng thời với $f(p) = 0$, 
các phương pháp Newton và Secant thường gặp khó khăn. 
Để xem xét chi tiết hơn vấn đề này, ta đưa ra định nghĩa sau.

\subsection*{\textbf{Định nghĩa 2.10 (Nghiệm bội)}}

Một nghiệm $p$ của phương trình $f(x) = 0$ được gọi là 
\textit{nghiệm bội (zero of multiplicity)} bậc $m$ của $f$ 
nếu, với mọi $x \neq p$, ta có thể viết
\[
f(x) = (x - p)^m q(x),
\]
trong đó 
\[
\lim_{x \to p} q(x) \neq 0.
\]

Với đa thức, $p$ là nghiệm bội bậc $m$ của $f$ 
nếu $f(x) = (x - p)^m q(x)$ và $q(p) \neq 0$.

Về bản chất, $q(x)$ biểu diễn phần của hàm $f(x)$ 
không đóng góp vào nghiệm tại $x = p$. 
Kết quả tiếp theo sẽ giúp xác định các \textit{nghiệm đơn (simple zeros)}, 
tức là những nghiệm có bội bằng $1$.

\subsection*{\textbf{Định lý 2.11}}

Giả sử $f \in C^1[a,b]$. 
Hàm $f$ có \textit{nghiệm đơn (simple zero)} tại $p \in (a,b)$ 
khi và chỉ khi 
\[
f(p) = 0 \quad \text{và} \quad f'(p) \neq 0.
\]

\subsection*{\textbf{Định lý 2.12}}

Giả sử $f \in C^m[a,b]$.  
Hàm $f$ có nghiệm bội $m$ tại $p \in (a,b)$ 
khi và chỉ khi
\[
0 = f(p) = f'(p) = f''(p) = \cdots = f^{(m-1)}(p), 
\quad \text{nhưng} \quad f^{(m)}(p) \neq 0.
\]

Kết quả trong Định lý 2.12 cho thấy rằng 
nếu tồn tại một lân cận của $p$ mà tại đó phương pháp Newton được áp dụng, 
thì nó sẽ hội tụ bậc hai về $p$ đối với mọi giá trị khởi đầu $p_0$ đủ gần $p$, 
miễn là $p$ là nghiệm đơn.  
Ví dụ tiếp theo sẽ minh hoạ rằng hội tụ bậc hai có thể không xảy ra 
khi nghiệm không phải là nghiệm đơn.

\subsection*{Ví dụ 1 – Ảnh hưởng của nghiệm bội đến phương pháp Newton}

Xét hàm
\[
f(x) = e^x - x - 1.
\]

\begin{enumerate}[label=(\alph*)]
  \item Chứng minh rằng $f$ có nghiệm bội 2 tại $x = 0$.
  \item Áp dụng phương pháp Newton với $p_0 = 1$ để khảo sát tốc độ hội tụ.
\end{enumerate}

\textbf{Lời giải.}

(a)  
Ta có
\[
f'(x) = e^x - 1, 
\qquad 
f''(x) = e^x.
\]
Tại $x = 0$, ta nhận được
\[
f(0) = 0, 
\quad 
f'(0) = 0, 
\quad 
f''(0) = 1 \neq 0.
\]
Do đó, $x = 0$ là nghiệm bội 2 của $f(x)$.

(b)  
Áp dụng công thức Newton
\[
p_{n} = p_{n-1} - \frac{f(p_{n-1})}{f'(p_{n-1})}
       = p_{n-1} - \frac{e^{p_{n-1}} - p_{n-1} - 1}{e^{p_{n-1}} - 1}.
\]

Với $p_0 = 1$, ta thu được:
\[
\begin{aligned}
p_1 &= 1 - \frac{e - 2}{e - 1} 
      \approx 0.5819767, \\
p_2 &= 0.5819767 - 
      \frac{e^{0.5819767} - 0.5819767 - 1}{e^{0.5819767} - 1} 
      \approx 0.3190623, \\
p_3 &\approx 0.1729500, \\
p_4 &\approx 0.0906560, \\
p_5 &\approx 0.0460287.
\end{aligned}
\]

\begin{center}
\captionof{table}{Các giá trị lặp của phương pháp Newton cho $f(x)=e^x-x-1$}
\label{tab:newton-multroot}
\begin{tabular}{|c|c|}
\hline
$n$ & $p_n$ \\
\hline
0 & 1.0000000 \\
1 & 0.5819767 \\
2 & 0.3190623 \\
3 & 0.1729500 \\
4 & 0.0906560 \\
5 & 0.0460287 \\
\hline
\end{tabular}
\end{center}

Ta thấy dãy $\{p_n\}$ hội tụ đến nghiệm $p = 0$, 
nhưng tốc độ hội tụ chỉ là \textit{tuyến tính}, không phải bậc hai như trường hợp nghiệm đơn. 
Đây chính là hiện tượng chung của phương pháp Newton khi áp dụng cho các nghiệm bội.

