\section{Tăng tốc độ hội tụ(Accelerating Convergence)}

Định lý 2.8 cho thấy rằng hội tụ bậc hai là một tính chất hiếm khi đạt được. 
Phần này giới thiệu một kỹ thuật gọi là \textit{phương pháp Aitken’s $\Delta^2$}, 
dùng để \textit{tăng tốc quá trình hội tụ tuyến tính} của một dãy, 
bất kể dãy đó bắt nguồn từ đâu hay áp dụng trong trường hợp nào.

Nhà toán học Alexander Aitken (1895–1967) đã đề xuất kỹ thuật này năm 1926 
để tăng tốc hội tụ của các chuỗi trong nghiên cứu về phương trình đại số. 
Quá trình này tương tự với phương pháp đã được nhà toán học Nhật Bản 
Takakazu Seki Kowa (1642–1708) sử dụng từ thế kỷ 17.

---

\subsection*{\textbf{Phương pháp Aitken’s $\Delta^2$}}

Giả sử $\{p_n\}_{n=0}^\infty$ là dãy hội tụ tuyến tính đến giới hạn $p$.  
Để xây dựng một dãy hội tụ nhanh hơn, ta giả định rằng các sai số 
$(p_n - p)$, $(p_{n+1} - p)$ và $(p_{n+2} - p)$ có cùng dấu 
và $n$ đủ lớn sao cho
\[
\frac{p_{n+1} - p}{p_n - p} 
= \frac{p_{n+2} - p}{p_{n+1} - p}.
\]
Giải phương trình này theo $p$ thu được:
\[
p = p_n - \frac{(p_{n+1} - p_n)^2}{p_{n+2} - 2p_{n+1} + p_n}.
\]
Công thức trên định nghĩa \textit{một dãy mới} hội tụ nhanh hơn đến $p$ 
so với dãy gốc $\{p_n\}$.

\subsection*{Ví dụ 1}

Xét dãy $p_n = \cos(1/n)$, hội tụ tuyến tính đến $p = 1$.  
Dùng phương pháp Aitken’s $\Delta^2$ để tạo ra năm phần tử đầu tiên của dãy hội tụ nhanh hơn.

\textbf{Lời giải.}  
Để tính một phần tử $p_n^{(A2)}$, cần ba phần tử liên tiếp $p_n$, $p_{n+1}$, $p_{n+2}$ của dãy gốc.  
Do đó, để xác định $p_5^{(A2)}$, cần bảy phần tử đầu tiên của $\{p_n\}$.  
Bảng dưới đây cho thấy dãy $\{p_n^{(A2)}\}$ hội tụ đến $p = 1$ nhanh hơn rõ rệt so với $\{p_n\}$.

\begin{center}
\captionof{table}{So sánh tốc độ hội tụ của dãy gốc và dãy Aitken’s $\Delta^2$}
\label{tab:aitken-example}
\begin{tabular}{|c|c|c|}
\hline
$n$ & $p_n = \cos(1/n)$ & $p_n^{(A2)}$ \\
\hline
1 & 0.540302 & -- \\
2 & 0.877583 & -- \\
3 & 0.940007 & 0.998947 \\
4 & 0.968912 & 1.000020 \\
5 & 0.983347 & 1.000000 \\
6 & 0.991698 & 1.000000 \\
7 & 0.996043 & 1.000000 \\
\hline
\end{tabular}
\end{center}

\subsection*{\textbf{Định nghĩa 2.13 (Sai phân tiến)}}

Với dãy $\{p_n\}_{n=0}^\infty$, \textit{sai phân tiến (forward difference)} được định nghĩa là
\[
\Delta p_n = p_{n+1} - p_n, \quad n \ge 0.
\]
Sai phân bậc cao hơn được định nghĩa đệ quy:
\[
\Delta^k p_n = \Delta(\Delta^{k-1} p_n), \quad k \ge 2.
\]
Do đó,
\[
\Delta^2 p_n = p_{n+2} - 2p_{n+1} + p_n.
\]
Khi đó, công thức Aitken có thể được viết gọn dưới dạng:
\[
p_n^{(A2)} = p_n - \frac{(\Delta p_n)^2}{\Delta^2 p_n}.
\]


\subsection*{\textbf{Định lý 2.14}}

Giả sử $\{p_n\}$ hội tụ tuyến tính đến giới hạn $p$ và
\[
\lim_{n \to \infty} \frac{p_{n+1} - p}{p_n - p} = \lambda, \quad |\lambda| < 1.
\]
Khi đó, dãy Aitken $\{p_n^{(A2)}\}$ hội tụ đến $p$ \textit{nhanh hơn} dãy ban đầu, 
theo nghĩa:
\[
\lim_{n \to \infty} \frac{p_n^{(A2)} - p}{p_n - p} = 0.
\]

\subsection*{\textbf{Phương pháp Steffensen}}

Bằng cách áp dụng một biến thể của phương pháp Aitken’s $\Delta^2$ 
cho một dãy hội tụ tuyến tính thu được từ quá trình lặp điểm cố định, 
ta có thể tăng tốc độ hội tụ lên bậc hai. 
Thủ tục này được gọi là \textit{phương pháp Steffensen}, 
và chỉ khác một chút so với việc áp dụng trực tiếp công thức Aitken’s $\Delta^2$ 
cho dãy lặp điểm cố định hội tụ tuyến tính.

Phương pháp Aitken’s $\Delta^2$ xây dựng các phần tử theo thứ tự:
\[
p_0, \quad p_1 = g(p_0), \quad p_2 = g(p_1), \quad 
\hat{p}_0 = [\Delta^2](p_0),
\]
trong đó ký hiệu $[\Delta^2]$ biểu thị công thức (2.15) được sử dụng, tức là
\[
\hat{p}_0 = p_0 - \frac{(p_1 - p_0)^2}{p_2 - 2p_1 + p_0}.
\]

Tiếp tục quá trình, ta có
\[
p_3 = g(p_2), \quad \hat{p}_1 = [\Delta^2](p_1), \ \text{v.v.}
\]
Phương pháp Steffensen xây dựng các phần tử đầu tiên $p_0, p_1, p_2$ và $\hat{p}_0$ 
giống như phương pháp Aitken, nhưng sau đó sử dụng $\hat{p}_0$ 
làm giá trị khởi đầu mới thay vì $p_2$.  

Nói cách khác, ta có dãy được tạo thành:
\[
p_0^{(0)}, \quad 
p_1^{(0)} = g(p_0^{(0)}), \quad 
p_2^{(0)} = g(p_1^{(0)}), \quad 
p_0^{(1)} = [\Delta^2](p_0^{(0)}), \quad 
p_1^{(1)} = g(p_0^{(1)}), \ldots
\]

Cứ mỗi ba phần tử, một phần tử mới của dãy Steffensen được sinh ra bằng công thức Aitken’s $\Delta^2$, 
trong khi các phần tử còn lại được tính bằng phép lặp điểm cố định thông thường.  

\subsection*{\textbf{Thuật toán – Phương pháp Steffensen}}

INPUT:
Giá trị khởi đầu $p_0$, sai số cho phép \texttt{TOL}, 
và số lần lặp tối đa $N_0$.

OUTPUT:
Nghiệm gần đúng $p$, hoặc thông báo thất bại.

\begin{enumerate}
  \item Đặt $i = 1$.
  \item Trong khi $i \le N_0$, thực hiện các bước 3–6:
  \begin{enumerate}
    \item Tính $p_1 = g(p_0)$; 
    \item Tính $p_2 = g(p_1)$;
    \item Tính
    \[
    p = p_0 - \frac{(p_1 - p_0)^2}{p_2 - 2p_1 + p_0}.
    \]
    \item Nếu $|p - p_0| < \texttt{TOL}$, 
    thì xuất $p$ và \textbf{STOP} (thủ tục thành công).
  \end{enumerate}
  \item Đặt $i = i + 1$.
  \item Gán $p_0 = p$ (cập nhật giá trị khởi đầu).
  \item Nếu $i > N_0$, xuất thông báo: 
  ``Phương pháp thất bại sau $N_0$ lần lặp.'' và \textbf{STOP}.
\end{enumerate}

\textit{Ghi chú:}  
Nếu mẫu số $(p_2 - 2p_1 + p_0)$ bằng 0, 
thì công thức Aitken’s $\Delta^2$ không xác định. 
Trong trường hợp đó, thuật toán sẽ dừng và 
lấy $p_2$ làm xấp xỉ tốt nhất có thể.

\subsection*{Minh hoạ}

Để giải phương trình
\[
x^3 + 4x^2 - 10 = 0
\]
bằng \textit{phương pháp Steffensen}, ta viết lại
\[
x^3 + 4x^2 = 10,
\]
suy ra công thức lặp điểm cố định:
\[
g(x) = \sqrt{\frac{10}{x + 4}}.
\]

Phương pháp lặp điểm cố định này đã được xét trong Bảng~2.2, cột (d) của Mục~2.2.  
Áp dụng quy trình Steffensen với $p_0 = 1.5$, ta thu được các giá trị trong Bảng~\ref{tab:steffensen-iteration}.  
Giá trị $p_0^{(2)} = 1.365230013$ chính xác đến chữ số thập phân thứ 9.  

Trong ví dụ này, phương pháp Steffensen cho kết quả chính xác tương đương 
với phương pháp Newton, \textit{nhưng không cần tính đạo hàm}.  
Kết quả tương tự có thể quan sát ở phần minh hoạ cuối của Mục~2.2.

\begin{center}
\captionof{table}{Các bước lặp của phương pháp Steffensen cho $f(x) = x^3 + 4x^2 - 10$}
\label{tab:steffensen-iteration}
\begin{tabular}{|c|c|c|c|}
\hline
$k$ & $p_0^{(k)}$ & $p_1^{(k)} = g(p_0^{(k)})$ & $p_2^{(k)} = g(p_1^{(k)})$ \\
\hline
0 & 1.5 & 1.348399725 & 1.367376372 \\
1 & 1.365265224 & 1.365225534 & 1.365230583 \\
2 & 1.365230013 & -- & -- \\
\hline
\end{tabular}
\end{center}

\subsection*{\textbf{Định lý 2.15 (Hội tụ bậc hai của phương pháp Steffensen)}}

Giả sử $x = g(x)$ có nghiệm $p$ với $g'(p) \neq 0$.  
Nếu tồn tại $\delta > 0$ sao cho $g \in C^3[p - \delta,\, p + \delta]$, 
thì phương pháp Steffensen hội tụ \textit{bậc hai} về $p$ 
cho mọi giá trị khởi đầu $p_0 \in [p - \delta,\, p + \delta]$.



