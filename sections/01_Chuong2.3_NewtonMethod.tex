\section{Phương pháp Newton và các mở rộng}

Isaac Newton (1643–1727) là một trong những nhà khoa học vĩ đại nhất mọi thời đại. 
Vào cuối thế kỷ 17, các công trình của ông đã chạm đến hầu hết các lĩnh vực toán học. 
Phương pháp mang tên ông được giới thiệu nhằm tìm nghiệm của phương trình
\[
y^3 - 2y - 5 = 0.
\]
Mặc dù Newton minh hoạ phương pháp chủ yếu cho đa thức, ông đã nhận ra khả năng ứng dụng rộng hơn của nó.

Phương pháp Newton (Newton--Raphson) là một trong những kỹ thuật số mạnh mẽ và nổi tiếng để tìm nghiệm của phương trình phi tuyến. 
Có nhiều cách trình bày phương pháp: (i) trực quan bằng hình học/đồ thị như trong giải tích; 
(ii) như một lược đồ có tốc độ hội tụ nhanh hơn so với một số lặp hàm khác; 
và (iii) dựa trên khai triển Taylor, vừa dẫn ra công thức, vừa cho phép ước lượng sai số xấp xỉ. 
Dưới đây là cách suy ra công thức bằng khai triển Taylor.

Giả sử $f \in C^2[a,b]$. Cho $p_0 \in [a,b]$ là một xấp xỉ ban đầu của nghiệm $p$ sao cho $f'(p_0) \neq 0$ và $|p - p_0|$ ``nhỏ''. 
Khai triển Taylor bậc nhất của $f$ tại $x=p_0$, đánh giá ở $x=p$, cho
\[
f(p) \;=\; f(p_0) + (p - p_0)\,f'(p_0) + \tfrac{1}{2} f''(\xi)\,(p - p_0)^2,
\]
với một $\xi$ nằm giữa $p$ và $p_0$. 
Vì $f(p)=0$, ta được
\[
0 \;=\; f(p_0) + (p - p_0)\,f'(p_0) + \tfrac{1}{2} f''(\xi)\,(p - p_0)^2.
\]
Nếu $|p - p_0|$ đủ nhỏ, bỏ qua hạng bậc hai cho xấp xỉ tuyến tính
\[
0 \;\approx\; f(p_0) + (p - p_0)\,f'(p_0),
\]
từ đó
\[
p \;\approx\; p_0 - \frac{f(p_0)}{f'(p_0)}.
\]

Suy ra công thức truy hồi Newton
\[
p_{n} \;=\; p_{n-1} - \frac{f(p_{n-1})}{f'(p_{n-1})}, \qquad n \ge 1. \tag{2.7}
\]

% (Tuỳ chọn) Hình minh hoạ tiếp tuyến:
% \begin{figure}[h!]
% \centering
% \includegraphics[width=0.6\textwidth]{figure_newton_tangent.png}
% \caption{Minh hoạ phương pháp Newton: từ $p_0$, lấy giao điểm trục hoành của tiếp tuyến tại $(p_0, f(p_0))$ để được $p_1$, rồi lặp.}
% \label{fig:newton-geom}
% \end{figure}

\subsection*{\textbf{Thuật toán – Phương pháp Newton}}

INPUT: giá trị gần đúng ban đầu $p_0$, sai số cho phép \texttt{TOL}, số lần lặp tối đa $N_0$.  

OUTPUT: nghiệm gần đúng $p$, hoặc thông báo thất bại.

\begin{enumerate}
  \item Đặt $i = 1$.
  \item Trong khi $i \leq N_0$, thực hiện:
  \begin{enumerate}
    \item Tính
    \[
    p = p_0 - \frac{f(p_0)}{f'(p_0)}.
    \]
    \item Nếu $|p - p_0| < \texttt{TOL}$, thì xuất $p$ và \textbf{STOP}.
    \item Đặt $i = i+1$, $p_0 = p$.
  \end{enumerate}
  \item Nếu chưa hội tụ sau $N_0$ bước, xuất thông báo: ``Phương pháp thất bại sau $N_0$ lần lặp.'' và \textbf{STOP}.
\end{enumerate}

\subsection*{Điều kiện dừng và dạng lặp của phương pháp Newton}

Các bất đẳng thức dừng được dùng trong phương pháp chia đôi cũng có thể áp dụng cho phương pháp Newton. 
Cụ thể, chọn một ngưỡng sai số $\varepsilon > 0$ và xây dựng dãy $\{p_n\}$ cho đến khi đạt một trong các điều kiện:

\[
|p_N - p_{N-1}| < \varepsilon \tag{2.8}
\]

\[
\frac{|p_N - p_{N-1}|}{|p_N|} < \varepsilon, 
\quad p_N \neq 0 \tag{2.9}
\]

hoặc

\[
|f(p_N)| < \varepsilon. \tag{2.10}
\]

Một dạng của (2.8) thường được sử dụng ở bước 4 trong Thuật toán 2.3. 
Lưu ý rằng không bất đẳng thức nào trong (2.8)--(2.10) cung cấp thông tin chính xác về sai số thực sự $|p_N - p|$.

Phương pháp Newton cũng có thể được xem như một dạng lặp điểm cố định, với công thức
\[
p_n = g(p_{n-1}) = p_{n-1} - \frac{f(p_{n-1})}{f'(p_{n-1})}, \quad n \geq 1. \tag{2.11}
\]

Đây chính là dạng lặp đã mang lại sự hội tụ rất nhanh được quan sát ở cột (e) trong Bảng~\ref{tab:fixedpoint-table}. 
Tuy nhiên, phương pháp không thể tiếp tục nếu $f'(p_{n-1}) = 0$ tại một bước nào đó. 
Trên thực tế, phương pháp Newton hiệu quả nhất khi $f'$ không tiến gần 0 trong một lân cận của nghiệm $p$.

\subsection*{\textbf{Ví dụ 1}}

Xét hàm $f(x) = \cos x - x$. Hãy xấp xỉ nghiệm của $f$ bằng:
\begin{enumerate}[label=(\alph*)]
  \item phương pháp lặp điểm cố định;
  \item phương pháp Newton.
\end{enumerate}

\textbf{Lời giải.}

(a) Nghiệm của bài toán tìm nghiệm cũng chính là nghiệm của bài toán điểm cố định $x = \cos x$. 
Đồ thị trong Hình~\ref{fig:fixedpoint-cosx} cho thấy có đúng một điểm cố định $p$ trong khoảng $[0,\tfrac{\pi}{2}]$. 

Với giá trị khởi đầu $p_0 = \pi/4$, ta thu được kết quả ở Bảng~\ref{tab:fixedpoint-cosx}.

\begin{center}
\captionof{table}{Lặp điểm cố định cho $f(x) = \cos x - x$, $p_0 = \pi/4$}
\label{tab:fixedpoint-cosx}
\begin{tabular}{|c|c|}
\hline
$n$ & $p_n$ \\
\hline
0 & 0.7853981635 \\
1 & 0.7071067810 \\
2 & 0.7602445972 \\
3 & 0.7246674808 \\
4 & 0.7487198858 \\
5 & 0.7325608446 \\
6 & 0.7434642113 \\
7 & 0.7361282565 \\
\hline
\end{tabular}
\end{center}

Từ bảng trên, có thể kết luận nghiệm $p \approx 0.74$.

(b) Để áp dụng phương pháp Newton, ta có
\[
f'(x) = -\sin x - 1.
\]

Với $p_0 = \pi/4$, lần lượt tính được:
\[
p_1 = p_0 - \frac{\cos(\pi/4) - \pi/4}{-\sin(\pi/4) - 1} \approx 0.7395361337,
\]
\[
p_2 = p_1 - \frac{\cos(p_1) - p_1}{-\sin(p_1) - 1} \approx 0.7390851781,
\]
\[
p_3 = p_2 - \frac{\cos(p_2) - p_2}{-\sin(p_2) - 1} \approx 0.7390851332.
\]

Kết quả được trình bày trong Bảng~\ref{tab:newton-cosx}.

\begin{center}
\captionof{table}{Phương pháp Newton cho $f(x) = \cos x - x$, $p_0 = \pi/4$}
\label{tab:newton-cosx}
\begin{tabular}{|c|c|}
\hline
$n$ & $p_n$ \\
\hline
0 & 0.7853981635 \\
1 & 0.7395361337 \\
2 & 0.7390851781 \\
3 & 0.7390851332 \\
4 & 0.7390851332 \\
\hline
\end{tabular}
\end{center}

Chỉ sau ba bước lặp, nghiệm gần đúng đã đạt $p \approx 0.7390851332$, 
chính xác hơn nhiều so với bảy bước lặp điểm cố định.

\subsection*{Hội tụ khi dùng phương pháp Newton}

Ví dụ 1 đã cho thấy phương pháp Newton có thể tạo ra xấp xỉ cực kỳ chính xác chỉ với rất ít bước lặp. 
Trong ví dụ đó, chỉ một lần lặp Newton đã cho độ chính xác tốt hơn bảy lần lặp bằng phương pháp điểm cố định. 
Bây giờ, ta sẽ xem xét kỹ hơn để hiểu tại sao Newton lại hiệu quả như vậy.

Khai triển Taylor ở đầu mục 2.3 đã chỉ ra tầm quan trọng của việc chọn gần đúng ban đầu chính xác. 
Giả định then chốt là số hạng bậc hai $(p - p_0)^2$ nhỏ hơn nhiều so với $|p - p_0|$, 
nên có thể bỏ qua. Điều này rõ ràng không đúng trừ khi $p_0$ đủ gần nghiệm $p$. 
Nếu $p_0$ không đủ gần nghiệm thực sự, thì khó có lý do để tin rằng Newton sẽ hội tụ về nghiệm. 
Tuy nhiên, trong một số trường hợp, ngay cả những xấp xỉ ban đầu kém cũng có thể dẫn đến hội tụ 
(xem thêm các Bài tập 15 và 16).

Định lý sau đây làm sáng tỏ sự hội tụ của phương pháp Newton và nhấn mạnh tầm quan trọng 
của việc chọn giá trị khởi đầu $p_0$.

\subsection*{Định lý (Hội tụ của phương pháp Newton)}

Giả sử $f \in C^2[a,b]$. Nếu $p \in (a,b)$ thỏa mãn $f(p)=0$ và $f'(p) \neq 0$, 
thì tồn tại một $\delta > 0$ sao cho phương pháp Newton sinh ra dãy 
$\{p_n\}_{n=1}^\infty$ hội tụ về $p$ với mọi giá trị khởi đầu 
$p_0 \in [p-\delta,\, p+\delta]$.

\textbf{Phác thảo chứng minh.}  
Xem phương pháp Newton như một lặp điểm cố định:
\[
p_n = g(p_{n-1}), \quad n \geq 1,
\]
trong đó
\[
g(x) = x - \frac{f(x)}{f'(x)}.
\]

Ta cần tìm một khoảng $[p-\delta,\,p+\delta]$ sao cho $g$ ánh xạ khoảng này vào chính nó 
và tồn tại hằng số $k < 1$ sao cho $|g'(x)| \leq k$ trên đó.

Vì $f'(p) \neq 0$ và $f \in C^2$, nên $g$ khả vi và liên tục trong một lân cận của $p$. 
Hơn nữa,
\[
g'(p) = \frac{f(p)f''(p)}{[f'(p)]^2} = 0,
\]
vì $f(p) = 0$. Do đó, tồn tại $\delta > 0$ sao cho $|g'(x)| < k < 1$ 
với mọi $x \in [p-\delta,\,p+\delta]$.

Theo Định lý điểm cố định, dãy $\{p_n\}$ sẽ hội tụ đến điểm cố định $p$.

\section*{\textbf{The Secant Method}}

Phương pháp Newton là một kỹ thuật rất mạnh, nhưng có một nhược điểm lớn: 
cần biết giá trị của đạo hàm $f'(x)$ tại mỗi bước lặp. 
Trong nhiều trường hợp, việc tính đạo hàm còn khó khăn và tốn nhiều phép tính hơn so với tính $f(x)$.

Để khắc phục vấn đề này, ta xét một biến thể: thay vì sử dụng $f'(p_{n-1})$, 
ta xấp xỉ nó bằng công thức sai phân
\[
f'(p_{n-1}) \;\approx\; \frac{f(p_{n-1}) - f(p_{n-2})}{p_{n-1} - p_{n-2}}.
\]

Thay vào công thức Newton, ta thu được
\[
p_n \;=\; p_{n-1} - f(p_{n-1}) \cdot \frac{p_{n-1} - p_{n-2}}{f(p_{n-1}) - f(p_{n-2})}, 
\qquad n > 1. \tag{2.12}
\]

Kỹ thuật này được gọi là \textit{phương pháp Secant} và được mô tả trong Thuật toán 2.4.

Phương pháp bắt đầu với hai giá trị khởi đầu $p_0$ và $p_1$. 
Nghiệm gần đúng $p_2$ chính là giao điểm với trục hoành của đường secant đi qua hai điểm 
$(p_0, f(p_0))$ và $(p_1, f(p_1))$. 
Sau đó, $p_3$ được tính bằng giao điểm của secant nối $(p_1, f(p_1))$ và $(p_2, f(p_2))$, 
và tiếp tục như vậy. 

Điểm quan trọng là, sau khi tính $f(p_0)$ và $f(p_1)$, 
mỗi bước của phương pháp Secant chỉ cần thêm một phép tính giá trị hàm số. 
Trong khi đó, mỗi bước của Newton yêu cầu cả $f$ và $f'$. 
Vì vậy, trong nhiều trường hợp phương pháp Secant là một lựa chọn kinh tế hơn.

---

\subsection*{\textbf{Thuật toán – Phương pháp Secant}}

INPUT: hai giá trị khởi đầu $p_0, p_1$, sai số cho phép \texttt{TOL}, số lần lặp tối đa $N_0$.  

OUTPUT: nghiệm gần đúng $p$, hoặc thông báo thất bại.

\begin{enumerate}
  \item Đặt $i = 2$.
  \item Trong khi $i \leq N_0$, thực hiện:
  \begin{enumerate}
    \item Tính
    \[
    p = p_{i-1} - f(p_{i-1}) \cdot \frac{p_{i-1} - p_{i-2}}{f(p_{i-1}) - f(p_{i-2})}.
    \]
    \item Nếu $|p - p_{i-1}| < \texttt{TOL}$, thì xuất $p$ và \textbf{STOP}.
    \item Đặt $i = i+1$, $p_{i-2} = p_{i-1}$, $p_{i-1} = p$.
  \end{enumerate}
  \item Nếu chưa hội tụ sau $N_0$ bước, xuất thông báo: 
  ``Phương pháp thất bại sau $N_0$ lần lặp.'' và \textbf{STOP}.
\end{enumerate}

\subsection*{\textbf{Phương pháp dây cung}}

Phương pháp Secant có thể hội tụ nhanh hơn phương pháp Newton trong một số trường hợp, 
nhưng nó không đảm bảo giữ nghiệm xấp xỉ trong khoảng ban đầu $[a,b]$. 
Nếu $f(a)$ và $f(b)$ trái dấu, phương pháp dây cung (\textit{method of false position}, 
hay \textit{Regula Falsi}) vừa tận dụng ý tưởng secant vừa giữ được tính chất 
“bracketing” như phương pháp chia đôi.

Ý tưởng cơ bản là: bắt đầu với $a_0$ và $b_0$ sao cho $f(a_0)f(b_0) < 0$. 
Xác định $p_0$ là giao điểm trục hoành của đường secant qua $(a_0,f(a_0))$ và $(b_0,f(b_0))$, tức là
\[
p_0 = b_0 - f(b_0)\,\frac{b_0 - a_0}{f(b_0) - f(a_0)}.
\]

Nếu $f(p_0) = 0$ thì $p_0$ chính là nghiệm. 
Nếu không, ta chọn cặp khoảng mới $[a_1,b_1]$ sao cho $f(a_1)f(b_1)<0$, 
trong đó một đầu mút là $p_0$ và đầu mút còn lại được giữ từ bước trước. 
Tiếp tục quá trình để thu được dãy $\{p_n\}$.

\textbf{Thuật toán – Phương pháp dây cung}

INPUT: $a$, $b$ sao cho $f(a)f(b)<0$; sai số cho phép \texttt{TOL}; số lần lặp tối đa $N_0$.  

OUTPUT: nghiệm gần đúng $p$, hoặc thông báo thất bại.

\begin{enumerate}
  \item Đặt $i = 1$.
  \item Trong khi $i \leq N_0$, thực hiện:
  \begin{enumerate}
    \item Tính
    \[
    p = b - f(b)\,\frac{b - a}{f(b) - f(a)}.
    \]
    \item Nếu $|f(p)| < \texttt{TOL}$, thì xuất $p$ và \textbf{STOP}.
    \item Nếu $f(a)f(p) < 0$, đặt $b = p$; ngược lại, đặt $a = p$.
    \item Đặt $i = i+1$.
  \end{enumerate}
  \item Nếu chưa hội tụ sau $N_0$ bước, xuất thông báo: 
  ``Phương pháp thất bại sau $N_0$ lần lặp.'' và \textbf{STOP}.
\end{enumerate}






